\documentclass[20pt,margin=1in,innermargin=-4.5in,blockverticalspace=-0.25in]{tikzposter}
\geometry{paperwidth=42in,paperheight=30in}
\usepackage[utf8]{inputenc}
\usepackage{amsmath}
\usepackage{amsfonts}
\usepackage{amsthm}
\usepackage{amssymb}
\usepackage{mathrsfs}
\usepackage{graphicx}
\usepackage{adjustbox}
\usepackage{enumitem}
\usepackage[backend=biber,style=numeric]{biblatex}
\usepackage{emory-theme}
\usepackage[scaled]{helvet}
\usepackage[T1]{fontenc}
\usepackage{tikz}
\usetikzlibrary{shapes.geometric,calc}


\usepackage{mwe} % for placeholder images

\addbibresource{refs.bib}

% set theme parameters
\tikzposterlatexaffectionproofoff
\usetheme{EmoryTheme}
\usecolorstyle{EmoryStyle}

\title{Parallelization in Multiple Imputation}
\author{Sven Nekula, Joshua Simon and Eva Wolf}
\institute{Otto Friedrich University, Bamberg}
\titlegraphic{\includegraphics[width=0.06\textwidth]{img/uni_ba_logo_100_blau.png}}

\newcommand\score[2]{%
	\pgfmathsetmacro\pgfxa{#1 + 1}%
	\tikzstyle{scorestars}=[star, star points=5, star point ratio=5, draw, inner sep=1.3pt, anchor=outer point 3]%
	\begin{tikzpicture}[baseline]
		\foreach \i in {1, ..., #2} {
			\pgfmathparse{\i<=#1 ? "yellow" : "gray"}
			\edef\starcolor{\pgfmathresult}
			\draw (\i*1.75ex, 0) node[name=star\i, scorestars, fill=\starcolor]  {};
		}
	\end{tikzpicture}%
}

% begin document
\renewcommand\familydefault{\sfdefault}
\begin{document}
\maketitle
\centering
\begin{columns}
    \column{0.32}
    \block{What is Parallelization?}{
         Parallelization is a technique to fasten time-consuming computations. It uses all the cores on a CPU (Central Processing Unit) parallely and splits up the computational work on them. Afterwords, the results are merged. This can reduce the time needed for a task. 
                 
         \begin{tikzfigure}[4 CPUs with 4 cores each]
         	\includegraphics[width=0.5\linewidth]{img/cpu.png}
         \end{tikzfigure}
    }
    \block{Implementations in R}{
     \textbf{foreach::foreach} usability \hspace{-1mm}\small{\score{2.4}{5}}\\
    \\
     \textbf{parlmice} 
    
     \textbf{micemd::mice.par}
      
     \textbf{parallel::parLapply}
          
     \textbf{purrr::future\_map} 
  }
	\block{Theory}{
	}
	\block{Methodology}{
	\textbf{time measurement} was done with the \textit{system.time} function, which returns 3 values: User CPU-, User System- and Elapsed time. User CPU is the time needed by the current task such as an execution in R. System CPU describes the time needed by the operating system to organize that task such as opening folders or asking for the System time. Elapsed time is the Wall Clock time that passed while the function was running.\\
	\textbf{speed up} \\
	}
    \column{0.36}
    \block{Results}{
        
        \vspace{1em}
        \begin{tikzfigure}[Runtime and Speedup from 1 up to 8 cores]
            \includegraphics[width=1\linewidth]{img/2022-01-11_benchmark_core_Linux_both}
        \end{tikzfigure}
        \vspace{1em}
       
    }

    \column{0.32}
    \block{Comparison}{
        Recent developments in symbolic group theory \cite{cite:0} have raised the question of whether $\mathscr{{J}} \le I$. The groundbreaking work of Q. Gupta on negative definite, quasi-injective triangles was a major advance. Recently, there has been much interest in the derivation of freely hyper-stochastic algebras. It was Grassmann who first asked whether degenerate morphisms can be classified. In \cite{cite:4}, the main result was the derivation of sub-analytically degenerate classes. Unfortunately, we cannot assume that $\mathfrak{{\ell}} ( \mathfrak{{z}}' ) \ne \| {\varepsilon_{\xi}} \|$.
        
        \begin{tikzfigure}[Look, my method is better.]
            \includegraphics[width=0.5\linewidth]{example-image}
        \end{tikzfigure}
    }
    
    \block{Application}{
      \textbf{disk framing} 
    }
    
    \block{References}{
        \vspace{-1em}
        \begin{footnotesize}
        \printbibliography[heading=none]
        \end{footnotesize}
    }
\end{columns}
\end{document}